\subsection{Radiative decays}

Rare radiative b-hadrons decays, probing the FCNC $b\to s \gamma$ transition, are complementary 
to $b\to s \ell \ell $ processes and are particularly sensitive to the $C_7$ Wilson coefficient. 

The LHCb core physics program includes also radiative $b$-mesons decays. 
In particular the LHCb experiment has published the first observation of the photon polarisation in $b\to s \gamma$
transitions through the study of $B^+ \to K^+ \pi^+ \pi^-\gamma$ decays~\cite{Aaij:2014wgo}. 
Furthermore  the ratio of branching fractions of the $B^0\to K^{\ast 0} \gamma$ and $B^0_s\to \phi \gamma$ decays has been measured
together with a search for direct CP asymmetry of the $B^0\to K^{\ast 0}\gamma$ decay~\cite{Aaij:2012ita}.
These measurements are only the starting point of a long program of measurements of which a notable example will be the measurement of the effective lifetime of the $B^0_s \to \phi\gamma$ decay, which is strongly sensitive to NP. 

The study of radiative $b$-hadron decays at ATLAS and CMS is difficult due to the low efficiency of reconstructing very low $p_T$ photons, either directly or as $e^+e^-$ pairs.
Therefore, at present, no contribution from the two experiments is foreseen in this field. 



