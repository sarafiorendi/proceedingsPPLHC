\subsection{Rare charm decays}

Rare decays involving charm hadrons are fundamental probes of new physics: 
not only being complementary to $b$- and $s$-hadrons by probing the up sector, 
but also because FCNC are much more suppressed in charm due to the lack of a high mass down-type quark. 

A rich program of rare charm physics is brought forward by LHCb, ranging from FCNC to the already mentioned LFV decays. 
In particular the studies of $D^0\to \pi^+ \pi^- \mu^+\mu^-$~\cite{Aaij:2013uoa}, 
$D^0\to K^- \pi^+ \mu^+\mu^-$~\cite{Aaij:2015hva}, $D^+_{(s)}\to \pi^+ \mu^+\mu^-$~\cite{Aaij:2013sua} and $D^0\to \mu^+\mu^-$~\cite{Aaij:2013cza}
are putting strong bounds on new physics scenarios, e.g. models with Leptoquarks~\cite{Bauer:2015knc}.
The $D^0 \to \mu^+ \mu^-$ decays has been studied also by the CMS experiment~\cite{Pedrini:2012vp}, however this was based on a loose trigger
configuration that could not be maintained.

The LHCb experiment will soon exploit Run II data and the study of rare charm decays will stay part of the core physics program
also in the upgrade phase. 
While ATLAS and CMS could explore the study of these decays exploiting events triggered independently by other particles, they are not expected to 
be competitive in the future on this field. 
The Belle II experiment is expected to produce significant result in this field, but we do not discuss futher this aspect here. 

