%\documentclass[11pt]{amsart}
%\usepackage{geometry}                % See geometry.pdf to learn the layout options. There are lots.
%\geometry{letterpaper}                   % ... or a4paper or a5paper or ... 
%%\geometry{landscape}                % Activate for for rotated page geometry
%%\usepackage[parfill]{parskip}    % Activate to begin paragraphs with an empty line rather than an indent
%\usepackage{graphicx}
%\usepackage{amssymb}
%\usepackage{epstopdf}
%\DeclareGraphicsRule{.tif}{png}{.png}{`convert #1 `dirname #1`/`basename #1 .tif`.png}
%
%\title{Brief Article}
%\author{The Author}
%%\date{}                                           % Activate to display a given date or no date
%
%\begin{document}
%\maketitle
\section{Introduction}

Rare decays proceeding via flavor changing neutral currents (FCNC) only occur at loop order and beyond in the Standard Model (SM), being forbidden at tree level.
However, in SM extensions new particles can contribute at loop or tree level, resulting in a modification of the amplitude of the process, appearance of new sources of CP violation or change in the angular distribution of final-state particles.
Therefore, information about physics beyond the SM can be gathered by searching for possible deviations from the predictions on branching fractions or angular distributions of rare decay processes.

This article reviews some of the studies on rare $b$ and $c$ decays conducted during LHC RunI by the ATLAS\cite{ATLAS}, CMS\cite{CMS} and LHCb\cite{LHCb} experiments, and outlines the perspectives for Run II measurements.\\

The LHCb detector has been specifically designed to access the physics of heavy Flavors in the forward region.
It is characterized by an excellent particle identification provided by RICH detectors, and good impact parameter and momenta resolutions.
LHCb runs at a reduced instantaneous luminosity in order to limit the pileup effects.
Thanks to its forward geometry and highly efficient and flexible trigger, able to efficiently select muons, electrons, hadrons and photons, LHCb can access very low transverse momentum ranges.

The general purpose experiments ATLAS and CMS cover a rapidity range up to $|\eta | <$ 2.5.
Since the bulk of the b-quark production peaks at large rapidities, they are able to collect only a fraction of the produced B-hadrons.
Both experiments fully exploits the tracking devices located closest to the interaction point and the muon-detectors, the farthest out sub-detector. 
Their capabilities to measure rare decay properties are almost entirely driven by their excellent muon reconstruction and identification performance.
In fact, in both experiments heavy flavor analyses are mainly based on multi-muon triggers, which can collect signals down to low muon transverse momenta (few GeV). \\

All the three experiments already explored the two benchmark channels, $B_{s(d)} \to \mu^+\mu^-$ and $B^0 \to K*\mu^+\mu^-$ using Run I data (ATLAS contribution on the latter measurement is still to be published).
The increased amount of events that will be delivered by the LHC during Run II will allow to improve the precision on those measurements, along with offering the opportunity to access other very rare decay modes.
However, due to the high instantaneous luminosity, a significant limitation for the two omni-purpose experiments is represented by the increase in the trigger thresholds, which could reduce the reach of their heavy flavor program.

Both the ATLAS and the CMS experiments have been facing significant detector improvements to deal with the high rate and pileup environment of Run II. 

In particular, a new pixel layer, called Insertable B-Layer (IBL), has been installed in the ATLAS detector during the long shutdown of the LHC machine. 
The new detector, inserted at a radius of about 33.25 mm from the beam-pipe, will provide up to a factor of 2 improvement in the impact parameter resolution for low p$_T$ tracks.
The ATLAS muon system has also been updated with the installation of new chambers and the addition of a new Thin Gap Chamber (TGC) coincidence layer, which minimizes fakes in the high pseudorapidity region.

The CMS pixel detector will be renewed during the extended end-of-year shutdown 2016-2017. 
The so-called Phase1 Pixel foresees one additional barrel layer and one additional endcap disk, which will provide more robust tracking performance. 
It also includes a new improved readout chip, lower material budget, and the installation of more efficient cooling and powering systems.\\


The trigger systems of all the three experiments have been revised for Run II.

Both the two levels of the ATLAS trigger system have been upgraded.
The capabilities of the Level-1 trigger were improved in several ways.
In particular, additional inputs were added to the Level-1 Muon trigger and both the Level-1 Calorimeter and Muon triggers have been interfaced to a new topological trigger processor, currently under commissioning, which supplies information about the event topology. 
This additional information will help in suppressing the background rate while maintaining low thresholds on muon transverse momentum, providing significant improvements in the low p$_T$ physics performance.
Furthermore, the ATLAS experiment planned the installation of an hardware processor dedicated to tracking: the Fast TracKer (FTK) processor. 
The FTK is designed to perform a fast track reconstruction between the Level-1 and the High Level Trigger step and to provide a track reconstruction similar to the offline reconstruction.

The CMS experiment underwent a complete renovation of its hardware trigger (L1) in 2016.
Both the L1 Calorimeter and Muon triggers have been upgraded.
As far as the Muon trigger is concerned, data from the three muon detection systems (Drift Tubes, Resistive Plate Chambers and Cathode Strip Chambers) have been combined in order to obtain a higher efficiency and better rate reduction. 
More sophisticated algorithms and the possibility to use more complex topological requirements and invariant mass selections also contribute to maintaining low thresholds on muon momenta and affordable rate.

The LHCb software trigger has been completely revised. 
The trigger farm has been improved to be able to write to storage 12.5 kHz instead of 5 kHz.
The software trigger has been split in two separate instances, the first one running synchronous with the LHC collisions while the second stage running asynchronously. 
This split allows to perform a real-time calibration and alignment of the detector. 
The software trigger is thus able to select events based on a reconstruction with a quality almost identical to the offline processing.


%\section{}
%\subsection{}

%@book{CMS,
%      title         = "{Technical proposal}",
%      publisher     = "CERN",
%      collaboration = "CMS Collaboration",
%      address       = "Geneva",
%      series        = "LHC Tech. Proposal",
%      year          = "1994",
%      url           = "https://cds.cern.ch/record/290969",
%      note          = "Cover title : CMS, the Compact Muon Solenoid : technical
%                       proposal",
%}
%@book{,
%      title         = "{LHCb : Technical Proposal}",
%      publisher     = "CERN",
%      address       = "Geneva",
%      series        = "Tech. Proposal",
%      year          = "1998",
%      url           = "https://cds.cern.ch/record/622031",
%}
%@article{ATLAS,
%      author         = "Armstrong, W. W. and others",
%      title          = "{ATLAS: Technical proposal for a general-purpose p p
%                        experiment at the Large Hadron Collider at CERN}",
%      collaboration  = "ATLAS",
%      year           = "1994",
%      reportNumber   = "CERN-LHCC-94-43",
%      SLACcitation   = "%%CITATION = CERN-LHCC-94-43;%%"
%}
%\end{document}  