% Please use the skeleton file you have received in the
% invitation-to-submit email, where your data are already
% filled in. Otherwise please make sure you insert your
% data according to the instructions in PoSauthmanual.pdf
%\documentclass{PoS}
%
%\title{Rare and semi-rare decays at ATLAS}
%
%\ShortTitle{Rare decays at ATLAS}
%
%\author{\speaker{Umberto De Sanctis}\thanks{on behalf of the ATLAS Collaboration}\\
%        University of Sussex\\
%        E-mail: \email{umberto.de.sanctis@cern.ch}}
%
%%\author{Another Author\\
%%        Affiliation\\
%%        E-mail: \email{...}}
%
%\abstract{}
%
%\FullConference{16th International Conference on B-Physics at Frontier Machines\\
%		2-6 May 2016\\
%		Marseille, France}
%
%\newcommand{\theInvRho}{\frac{\varepsilon_{\mu^+\mu^-}}{\varepsilon_{J/\psi K^{\pm}}}}
%\newcommand{\pt}{$p_{\rm{T}}$}
%\newcommand{\gev}{\rm{GeV}}
%
%\begin{document}
\section{Discussion}
The searches for rare and semi-rare decays of $B$ and $D$-mesons are central in the $B$-physics programme of all three experiments. As for all $B$-physics searches, the leading role is taken by the LHCb collaboration, whose detector was designed specifically for this type of physics. Nevertheless, ATLAS and CMS collaborations can be competitive in the searches for rare processes, that are basically limited by the size of the datasets to be analysed. In fact, both experiments will take advantage of the larger integrated luminosity they will be able to collect during Run 2 LHC campaign. Hence we envisage that, for some specific decay such as the $B_{(s)}^0 \to \mu \mu$ or those related to the $b \to s l^+ l^-$ transitions (see Section 2 and Section 3), the three experiments will be able to work together to combine their results to fully profit of the statistical power of the available datasets. A good step in this direction has been already made by CMS and LHCb collaborations in the combination of the $B_{(s)}^0 \to \mu \mu$ analyses and also by the creation of the LHCC Heavy-Flavour group. We hope that these steps will be the beginning of a fruitful collaboration among the LHC experiments.
%\end{document}